\documentclass[a4]{article}
\usepackage{amsmath}
\usepackage{amssymb}
\usepackage{listings}
\usepackage{graphicx}
\usepackage{../../manual/manual}
\newcounter{lst}[section]

\renewcommand{\thelst}{\arabic{section}.\arabic{lst}}

\lstset{basicstyle=\footnotesize\sffamily}

\lstdefinestyle{easslisting}{basicstyle=\footnotesize\sffamily, mathescape=true, frame=tb,  numbers=right, numberstyle=\footnotesize, stepnumber=1, numbersep=-5pt, captionpos=b}

\lstdefinestyle{eass}{basicstyle=\sffamily, mathescape=true}

\lstnewenvironment{listing}[3]{
  \noindent        
  \refstepcounter{lst}         
  \label{code:#1}  
\begin{tabular}{p{.97\columnwidth}} \\ \hline  {\normalsize \textbf{Code  fragment \arabic{section}.\arabic{lst}} #2} \\  \end{tabular} 
\lstset{language=#3,          
  basicstyle=\footnotesize\sffamily,
%%%  basicstyle=\footnotesize,    
  xleftmargin=10pt,    
  mathescape=true,    
  frame=tb,    
  numbers=right,    
%%%  numberstyle=\tiny,     
  numberstyle=\footnotesize,     
  stepnumber=1,     
  numbersep=-5pt}}{}

%-- definition for Gwendolen --%

\lstdefinelanguage{Gwendolen}{%
    morekeywords={Plans,Initial,Beliefs,Goals,name,fof-parse,Rules,Belief,Reasoning},
    morecomment=[l]{//},
 literate= {<-}{{$\leftarrow$}{$\:$}}2
           {.B}{{${\cal B}$}}2
           {.G}{{${\cal G}$}}2
           {lnot}{{$\sim$}}2
           {assert_shared}{{$+_{\Sigma}$}}2
           {remove_shared}{{$-_{\Sigma}$}}2
           {(perform)}{}0
}



\makeindex

\author{Louise A. Dennis}

\title{AJPF Tutorial 1 -- The Property Specification Language}

\begin{document}
\maketitle
This is the first in a series of tutorials on the use of the \ajpf\ model checking program.  This tutorial covers the basics of configuring a model-checking run and writing properties in \ajpf's property specification language.

Files for this tutorial can be found in the \texttt{mcapl} distribution in the directory \texttt{src/examples/gwendolen/ajpf\_tutorials/tutorial1}.

The tutorials assume some familiarity with the basics of running Java programs either at the command line or in Eclipse and some familiarity with the syntax and semantics of Linear Temporal Logic.

\section{Setting up Agent Java Pathfinder}
Before you can run \ajpf\ it is necessary to set up your computer to use Java Pathfinder.  There are instructions for doing this in the \mcapl\ manual (which you can find in the \texttt{doc} directory of the distribution).

The key point is that you need to create a file called \texttt{.jpf/site.properties} in your home directory on the computer you are using.  In this file you need to put one line which assigns the path to the \mcapl\ distribution to the key \texttt{mcapl}.  For instance if you have your \mcapl\ distribution in your home directory as  folder called \texttt{mcapl} then \texttt{site.properties} should contain the line.

\begin{verbatim}
mcapl = ${user.home}/mcapl
\end{verbatim}

We strongly recommend that you also set up an environment variable, \texttt{\$AJPF\_HOME}, set it to the path to the \mcapl\ directory and add this to your \texttt{.bashrc} or equivalent start-up files..

\section{A Simple Model Checking Attempt}

To run \ajpf\ you need to run the program \texttt{gov.jpf.tool.RunJPF} which is contained in \texttt{lib/3rdparty/RunJPF.jar} in the \mcapl\ distribution.  Alternatively you can use the \texttt{run-JPF (MCAPL)} Run Configuration in Eclipse.

You need to supply a JPF Configuration file as an argument.  You will find a sample file in \texttt{src/examples/gwendolen/ajpf\_tutorials/lifterandmedic.jpf}.  If you run this you should see output like the following:

\begin{verbatim}
JavaPathfinder v7.0 (rev ${version}) - (C) RIACS/NASA Ames Research Center


====================================================== system under test
ail.util.AJPF_w_AIL.main("/Users/louiseadennis/Eclipse/mcapl/src/examples/gwendolen/ail_tutorials/tutorial1/answers/ex2.ail","/Users/louiseadennis/Eclipse/mcapl/src/examples/gwendolen/ajpf_tutorials/tutorial1/lifterandmedic.psl","1")

====================================================== search started: 02/02/15 08:16

====================================================== results
no errors detected

====================================================== statistics
elapsed time:       00:00:12
states:             new=96, visited=116, backtracked=212, end=3
search:             maxDepth=16, constraints hit=0
choice generators:  thread=1 (signal=0, lock=1, shared ref=0), data=95
heap:               new=761483, released=752805, max live=4582, gc-cycles=212
instructions:       122799365
max memory:         495MB
loaded code:        classes=316, methods=4705

====================================================== search finished: 02/02/15 08:16
\end{verbatim}

Note this will take several seconds to generate.  We will discuss in future tutorials how to get more detailed output from the model checker.

At the moment the key point is the fact that it states \texttt{no errors detected}.  This means that the property supplied to the model checker was true for this program.

\section{JPF Configuration Files}

\texttt{lifterandmedic.jpf} is a JPF Configuration file.  There are a large number of configuration options for JPF which you can find in the JPF documentation~\footnote{Currently to be found at \url{http://babelfish.arc.nasa.gov/trac/jpf}}.  We will only discuss a handful of these options.  If you open \texttt{lifterandmedic.jpf} you should see the following:

\newpage
\noindent\rule{\textwidth}{1pt}
\begin{small}
\begin{verbatim}
@using = mcapl

target = ail.util.AJPF_w_AIL
target.args = ${mcapl}/src/examples/gwendolen/ail_tutorials/tutorial1/answers/ex2.ail,${mcapl}/src/examples/gwendolen/ajpf_tutorials/tutorial1/lifterandmedic.psl,1

\end{verbatim}
\end{small}
\rule{\textwidth}{1pt}

We explain each line of this below.

\begin{sloppypar}
\begin{description}
\item[@using = mcapl] Means that the proof is using the home directory for \texttt{mcapl} that you set up in \texttt{.jpf/site.properties}.
\item[target = ail.util.AJPF\_w\_AIL] This is the Java file containing the main method for the program to be model checked.  By default when model checking a program implemented using the \ail, you should use \texttt{ail.util.AJPF\_w\_AIL} as the target.  For those who are familiar with running programs in the \ail, this class is very similar to \texttt{ail.mas.AIL} but with a few tweaks to set up and optimise model checking.
\item[target.args =...] This sets up the arguments to be passed to \texttt{ail.util.AJPF\_w\_AIL}.  \texttt{ail.util.AJPF\_w\_AIL} takes three arguments.  In the configuration file these all have to appear on one line, separated by commas (but \emph{no spaces}).  This means you can not see them all in the file print out above.  In order the arguments are:
\begin{enumerate}
\item The first is an \ail\ configuration file.  In this example the file is \texttt{\$\{mcapl\}/src/examples/gwendolen/ail\_tutorials/tutorial1/answers/ex2.ail} which is a configuration file for a multi-agent system written in the \gwendolen\ language.
\item The second argument is a file containing a list of properties in \ajpf's property specification language that can be checked.  In this example this file is \texttt{lifterandmedic.psl} in the directory for this tutorial.
\item The last argument is the name of the property to be checked, \texttt{1} in this case.
\end{enumerate}
\end{description}
\end{sloppypar}

\section{The Property Specification Language}

\paragraph{Syntax}
The syntax for property formul\ae\ $\phi$ is as follows, where
$\AILagent$ is an ``agent constant'' referring to a specific agent in
the system, and $f$ is a ground first-order atomic formula (although it may use $\_$, as in \prolog, to indicate variables which may match any value):
%
\begin{equation*}
\begin{array}{rl}
\phi ::= & \lbelief{\AILagent}{f} \mid \lgoal{\AILagent}{f} \mid
  \lactions{\AILagent}{f} \mid \lintention{\AILagent}{f} \mid
  \lintendtodo{\AILagent}{f} \mid
  \lpercept{f} \mid \\ &
%% \\
%%    &
%% \mid \phi \land \phi
  \phi \vee \phi \mid \phi \wedge \phi \mid \neg \phi
   \mid \phi \: \until \phi \mid \phi \release \phi \mid \sometime \phi \mid \always \phi
\end{array}
\end{equation*}
%
Here, $\lbelief{\AILagent}{f}$ is true if $\AILagent$ believes
$f$ to be true, $\lgoal{\AILagent}{f}$ is true if $\AILagent$ has a
goal to make $f$ true, and so on (with $\lactionsfunc$ representing
actions, $\lintentionfunc$ representing intentions, $\lintendtodofunc$ representing the intention to take an action, and $\lperceptfunc$
representing percepts, i.e., properties that are true in the environment).

The following representation of this syntax is used in \ajpf's property specification files:

\begin{equation*}
\begin{array}{rl}
\phi ::= & \texttt{B(}\AILagent\texttt{, }f\texttt{)} \mid 
\texttt{G(}\AILagent\texttt{, }f\texttt{)} \mid 
\texttt{D(}\AILagent\texttt{, }f\texttt{)} \mid 
\texttt{I(}\AILagent\texttt{, }f\texttt{)} \mid 
\texttt{ItD(}\AILagent\texttt{, }f\texttt{)} \mid 
\texttt{P(}f\texttt{)} \mid \sim \phi \\
\phi' ::= & \phi \mid 
  \phi' \: \texttt{|\!|} \: \phi' \mid 
  \phi' \: \texttt{\&} \: \phi' \mid 
   \phi' \: \texttt{U} \: \phi' \mid \phi' \: \texttt{R} \: \phi' \mid \texttt{<>} \phi' \mid \texttt{[]} \phi'
\end{array}
\end{equation*}

Note, in particular, that in property specification files ``not'' ($\sim$) must always appear in an innermost position next to one of the BDI agent properties such as $\texttt{B(}\AILagent\texttt{, }f\texttt{)}$.  

It is also possible to use $\phi \texttt{->} \psi$ as shorthand for $\neg \phi \vee \psi$ in property specification files.

\medskip

\paragraph{Semantics}
%\noindent
We summarise semantics of property formul\ae{}.
Consider a program, $P$, describing a multi-agent system and let
$\MAS$ be the state of the multi-agent system at one point in the run
of $P$.  $\MAS$ is a tuple consisting of the local states of the individual agents and of the environment. Let $\AILagent \in \MAS$ be the state of an agent in the $\MAS$ tuple at this point in the 
program execution. Then 
$$
\MAS \models_{MC} \lbelief{\AILagent}{f} \text{\quad iff\quad }
\AILagent \models  \lbelief{\AILagent}{f}
$$
where $\models$ is logical consequence as implemented by the agent
programming language. The semantics of $\lgoal{\AILagent}{f}$, $\lintention{\AILagent}{f}$ and $\lintendtodo{\AILagent}{f}$ similarly refer to internal implementations of the language interpreter~\footnote{We briefly cover the \gwendolen\ implementation in appendix~\ref{s:impl_bdi}.}.  
The interpretation of $\lactions{\AILagent}{f}$ is:
$$\MAS \models_{MC}
\lactions{\AILagent}{f}$$ if, and only if, the last action changing
the environment was action $f$ taken by agent $\AILagent$.  Finally, the interpretation of
$\lpercept{f}$ is given as:
$$\MAS
\models_{MC} \lpercept{f}$$ if, and only if, $f$ is a percept that
holds true in the environment.
\medskip

\noindent The other operators in the \ajpf{} property specification
language have standard PLTL semantics~\cite{emerson:90a} and are
implemented as B\"{u}chi Automata as described
in~\cite{Gerth:1995:SOA:645837.670574,Courcoubetis92mea}. Thus, the
classical logic operators are defined by:
$$
\begin{array}{rcl}
%%  \MAS \models_{MC} \varphi \land \psi &\quad\text{iff}\quad& \MAS
%%  \models_{MC} \varphi \text{ and } \MAS \models_{MC} \psi\\
  \MAS \models_{MC} \varphi \vee \psi &\text{iff}& \MAS \models_{MC}
  \varphi \text{ or } \MAS \models_{MC} \psi \\
  \MAS \models_{MC} \neg \phi &\text{iff}& \MAS \not\models_{MC} \phi.
\end{array}
$$
The temporal formul\ae\ apply to runs of the programs in the \jpf{}
model checker. A run consists of a (possibly infinite) sequence of
program states $\MAS_i$, $i \geq 0$ where $\MAS_0$ is the initial
state of the program (note, however, that for model checking the
number of \emph{different} states in any run is assumed to be
finite). Let $P$ be a multi-agent program, then:
$$
\begin{array}{lcl}
  \MAS \models_{MC}\quad \varphi \: \until \psi 
   &\quad\text{iff}\quad&
    \text{ in all runs of $P$ there exists a state } \MAS_j\\
  &&  \text{ such that } \MAS_i \models_{MC}\ \varphi \text{ for all } 0 \leq i < j \\
  && \text{ and } \MAS_j \models_{MC} \psi{}.\\[.5em]
%
  \MAS \models_{MC}\quad\varphi \release \psi  
   &\quad\text{iff}\quad&
    \text{ either } \MAS_i \models_{MC} \varphi \text{ for all } i \text{ or there }\\
  && \text{ exists } \MAS_j \text{ such  that } \MAS_i \models_{MC} \varphi\\
  && \text{ for all } 0 \leq i \leq j \text{ and } \MAS_j \models_{MC} \varphi \land \psi{}.
\end{array}
$$
%
Conjunction $\land$ and the common temporal operators $\sometime$ (eventually) and $\always$
(always) are, in turn, derivable from $\vee$, $\until$ and $\release$ in the
usual way~\cite{emerson:90a}.

\section{Exercises}
If you look in \texttt{lifterandmedic.psl} you should see the following:
\begin{verbatim}
1: [] (~B(medic, bad))
\end{verbatim}
So this file contains one formula, labelled, \texttt{1} and the formula is equivalent to $\always \neg \lbelief{\texttt{medic}}{bad}$ -- which means it is always the case the medic agent doesn't believe the formula \texttt{bad} (or alternatively that the medic agent never believes \emph{bad}.

As noted previously the multi-agent system in \texttt{ex2.ail} is a \gwendolen\ program and is, in fact, the one described in the 8th \gwendolen\ tutorial in the \mcapl\ distribution.  It isn't necessary, for this tutorial, to understand the implementation of the BDI modalities (belief, goal, intention etc.) in the \gwendolen\ interpreter but a brief discussion is included as an appendix.

Adapt the JPF configuration file and extend the property specification file in the tutorial directory in order to verify the following properties of the multi-agent system.  You can find sample answers in the \texttt{answers} directory.

\begin{enumerate}
\item Eventually the lifter believes $human(5, 5)$.
$$\sometime \lbelief{\texttt{medic}}{human(5, 5)}$$
\item Eventually the medic has the goal $assist\_human(5, 5)$.
$$\sometime \lgoal{\texttt{medic}}{human(5, 5)}$$
\item Eventually the lift believes $human(3, 4)$ and eventually the lifter believes $holding(rubble)$.
$$\sometime \lbelief{\texttt{lifter}}{human(3, 4)} \wedge \sometime \lbelief{\texttt{lifter}}{holding(rubble)}$$
\item If the lifter has the intention to $goto55then34$ then eventually the medic will have the goal $assist\_human(5, 5)$. 
$$\lintention{\texttt{lifter}}{goto55then34} \Rightarrow \lgoal{\texttt{medic}}{assist\_human(5, 5)}$$
\item It is always the case that if the lifter does $move\_to(5, 5)$ then $human(5, 5)$ becomes perceptible.  
$$\always (\lactions{\texttt{lifter}}{move\_to(5, 5)} \Rightarrow \lpercept{human(5, 5)}$$
\item Eventually the lifter intends to move to (5, 5).
$$\sometime \lintendtodo{\texttt{lifter}}{move\_to(5, 5)}$$
\item Eventually the lifter intends to send the medic a perform request to assist the human in some square.
$$\sometime \lintendtodo{\texttt{lifter}}{send(\texttt{medic}, 2, assist\_human(\_, \_))}$$
For this you should consult the section on intending to send messages in appendix~\ref{s:impl_bdi}.
\end{enumerate}

\bibliographystyle{abbrv} %% {plain}
\bibliography{../../manual/manual}

\appendix
\section{Implementation of BDI Modalities in \gwendolen}
\label{s:impl_bdi}

In \gwendolen\ the BDI modalities of the \ajpf\ property specification language are implemented as follows.

\begin{itemize}
\item $\lbelief{\AILagent}{f}$.  An agent, $\AILagent$, believes the formula, $f$, if $f$ appears in its belief base or is deducible from its goal base using its reasoning rules.
\item $\lgoal{\AILagent}{f}$.  An agent, $\AILagent$, has a goal $f$, if $f$ is a goal that appears in the agent's goal base.
\item $\lintention{\AILagent}{f}$.  An agent, $\AILagent$, has an intention $f$, if $f$ is a goal in the goal base a plan has been selected to achieve or perform the goal.
\item $\lintendtodo{\AILagent}{f}$.  An agent, $\AILagent$, intends to do $f$, if $f$ is an action that appears in the deed stack of some intention.
\end{itemize}

\subsection{Intending to Send a Message}
\gwendolen\ uses a special syntax for send actions (\texttt{.send(ag, :tell, c)}) which is not recognised by the property specification language.  If you want to check that a \gwendolen\ agent intends to send a messsage then you need to use the syntax \texttt{send(agname, number, c)} where \texttt{agname} is the name of the recipient, \texttt{number} is
\begin{description}
\item[1] For \texttt{:tell},
\item[2] For \texttt{:perform},
\item[3] For \texttt{:achieve}
\end{description}
and \texttt{c} is the content of the message.

\end{document}
